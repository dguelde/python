\documentclass[11pt, oneside]{article}   	% use "amsart" instead of "article" for AMSLaTeX format
\usepackage{geometry}                		% See geometry.pdf to learn the layout options. There are lots.
\geometry{letterpaper}                   		% ... or a4paper or a5paper or ... 
%\geometry{landscape}                		% Activate for rotated page geometry
%\usepackage[parfill]{parskip}    		% Activate to begin paragraphs with an empty line rather than an indent
\usepackage{graphicx}				% Use pdf, png, jpg, or eps§ with pdflatex; use eps in DVI mode
								% TeX will automatically convert eps --> pdf in pdflatex		
\usepackage{amssymb}
\usepackage[document]{ragged2e}

%SetFonts

%SetFonts


\title{Brief Article}
\author{The Author}
%\date{}							% Activate to display a given date or no date

\begin{document}
%\maketitle
%\section{}
%\subsection{}
\begin{flushright}
 Donovan Guelde\\
PS1\\
CSCI 5454\\
Collaborators: None\\
\end{flushright}

\justify
1.  To show $(n+a)^{b} = \Theta (n^{b})$, we must show both $(n+a)^{b} = O(n^{b})$ and $(n+a)^b= \Omega (n^{b})$\\
\\
\indent
$(n+a)^{b} = O(n^{b})$:\\
\indent\indent
There exists some constant $c$ s.t. $(n+a)^{b} \leq c(n^{b})$\\
\indent\indent\indent
if $a <  0$ then $(n+a) < n $, so $(n+a)^b < c(n^b)$ where $c\geq 1$\\
\indent\indent\indent
if $a = 0$ then $(n+a) = n$, so $(n+a)^b = c(n^b)$ where $c \geq 1$\\
\indent\indent\indent
if $a > 0$ then $(n+a) > n$, so we must find constant c s.t. $(n+a)^b \leq c(n^b)$:\\
\indent\indent\indent\indent
as n approaches infinity, $(n+a) \leq 2n$ because a is a constant, and regardless\\
\indent\indent\indent\indent of a's value, 2n will eventually be larger than $(n+a)$. So:\\
\indent\indent\indent\indent\indent
$(n+a)^b \leq (2n)^b$\\
\indent\indent\indent\indent\indent
Therefore, $(n+a)^b \leq c(n^b)$ where $c=2^b$\\
\indent\indent
Therefore, $(n+a)^b \leq c(n^b)$ for all  constants $c \geq 2.$ $(n+a)^b = O(n^b)$\\
\\
\indent $(n+a)^{b} = \Omega (n^{b})$:\\
\indent\indent
There exists some constant $c$ s.t. $(n+a)^{b} \geq c(n^{b})$\\
\indent\indent\indent
if $a = 0$ then $(n+a) = n$, so $(n+a)^b = c(n^b)$ where $c \geq 1$\\
\indent\indent\indent
if $a > 0$ then $(n+a) >n $, so $(n+a)^b > c(n^b)$ where $c\geq 1$\\
\indent\indent\indent
if $a < 0$ then $(n+a) < n$, so we must find constant c s.t. $(n+a)^b \geq c(n^b)$:\\
\indent\indent\indent\indent
as n approaches infinity, $(n+a) \geq \frac{n}{2}$ because a is a constant, and regardless\\
\indent\indent\indent\indent of a's value, $(n+a)$ will eventually be larger than $\frac{n}{2}$. So:\\
\indent\indent\indent\indent $(n+a)^b \geq c(n)^b$ for all c s.t. $c \leq (\frac{1}{2})^b$\\
\\
\indent As $(n+a)^b = \Omega (n^b)$ and $(n+a)^b = O(n^b)$, it follows that $(n+a)^b = \Theta (n^b)$\\
\\
2. $1 = (n^ {\frac{1}{n^{log n}}}) \leq (2^{log*n}) \leq  n \leq nlog(n) \leq n^2 \leq (\sqrt{2})^{log n} \leq (\frac{3}{2})^n$\\
\indent $notes$:\\
\indent 1 is, of course, linear\\
\indent $(n^ {\frac{1}{n^{log n}}})$ approaches 2 as n approaches infinity, so this expression is also linear\\
\indent $(2^{log*n})$ grows extremely slowly, but does grow faster than linearly. When $n = 10^6$,\\ \indent\indent $2^{log*n} = 16$ \\
\indent
$n$, $n log(n)$, and $n^2$ are all textbook examples of big O ordering.\\
\indent $(\sqrt{2})^{log n}$ and $(\frac{3}{2})^n$ both exhibit exponential growth, but since $n > log n$ for all $n>1$,\\
\indent\indent $(\frac{3}{2})^n$ grows faster than $(\sqrt{2})^{log n}$\\
\\
3.a. $T(n) = T(n-1) +n, T(1)=1$\\
\indent
Expanding the recursion results in:\\
\indent\indent
$T(n) = T(n-1) +n$\\
\indent\indent
$T(n-1) = T(n-2) + (n-1)$\\
\indent\indent
$T(n-2) = T(n-3) + (n-2)$\\
\indent\indent
$T(n-3) = T(n-4) + (n-3)$\\
\indent\indent
...\\
\indent\indent
$T(2) = T(1) + 2$\\
\indent\indent
$T(1)=1$\\ \\
\indent\indent It quickly becomes apparent that  $T(n) = T(n-1) +n$ expands to:\\
\indent\indent $\sum_{i}^{n}(i) = n^2$\\
\\
3.b. $T(n) = 2T(\frac{n}{2}) + n^3; T(1)=1$\\
\indent Using the Master Theorem, a = 2, b = 2, d = 3\\
\indent $log_b a = log_2 2 = 1 < d$, so the recurrence grows at $O(n^3)$\\
\\
4.a.  Given an unsorted array U where all $U[i] > 0$ and $max(U) = k$:\\
\indent\indent S = [0]*k \#initialize an array of zeroes of size k\\
\indent\indent for all i in U:\\
\indent\indent\indent S[i]+=1 \# for every number in the unsorted array, increment the array S at\\
\indent\indent\indent\indent\indent index [number] by 1\\
\indent\indent for all j in S:\\
\indent\indent\indent while $S[j] >0$,\\ 
\indent\indent\indent print j\\
\indent\indent\indent S[j] -=1\\
4.b.  By starting with the assumptions that all items to be sorted are between(or equal to)\\
\indent 1 and k, we can instantly ignore the possibility of numbers falling outside this range.\\
\indent  This allows us to set up a data structure of size k, rather than worrying about values\\
\indent  of unknown size. By doing so, we are no longer forced to make comparisons between \\
\indent values, so the $nlog(n)$ limit no longer applies, as we are not recursing or building a tree.\\
\\
5.a.  To differentiate good minions from bad, choose a minion, say $M_1$, and ask the other minions if $M_1$ is good or bad, while disregarding $M_1$'s opinion on the other minion.  This will result in $n-1$ opinions about the given minion.  Since good minions always tell the truth, and bad minions are unreliable (may lie, but not necessarily), if $M_1$ receives at least $\frac{n}{2} $ good votes, then $M_1$ must be good.  If $M_1$ does not receive $\frac{n}{2}$ good votes, repeat the process for $M_2$, $M_3$, etc. until a good minion is found.  Once a good minion is found, his opinion can be used to judge all others.\\
\\
\indent However, if $\frac{n}{2}$ or more minions are bad, it will be impossible to reliably identify a good minion, as long as the bad minions conspire to trick Gru.  If exactly $\frac{n}{2}$ minions are bad, they can conspire to ensure than no good minion is identified, since a good minion can only be voted as good by a maximum of $\frac{n}{2} - 1$ other good minions.  In fact, they could conspire to vote 'good' for one of their own, giving Gru the false impression that he has identified a good minion.  This holds if more than $\frac{n}{2}$ minions are bad.  However, if not all of the bad minions conspire to fool GRU and truthfully report a good minion, Gru may be able to make the determination, but it is impossible to guarantee, as long as there are at least $\frac{n}{2}$ bad minions.\\
\\
5.b.  Given that at least $\frac{n}{2}$ minions are good and will always tell the truth:\\
\indent Choose $\lfloor \frac{n}{2} \rfloor$ minions at random and conduct pairwise comparisons with all other minions.  The results will fall into one of two categories:\\
\indent\indent (1) number of "Both Good" results $ \geq \lfloor \frac{n}{2} \rfloor$. Since at least $\frac{n}{2}$ minions are good, the only way to have $ \geq \lfloor \frac{n}{2} \rfloor$ "Both Good" results is if the minions chosen for the pairwise comparison testing are good themselves.\\
\indent\indent (2)  number of "Both Good" results $ < \lfloor \frac{n}{2} \rfloor$.  Since good minions outnumber bad, any result in this category indicates that the chosen minion is bad, even if the other bad minions lie and indicate the chosen minion is good.\\
\\
5.c.  Assuming that at least $\frac{n}{2}$ minions are good and will always tell the truth:\\
\indent 1.  Choose any minion, $M'$ at random, and conduct pairwise testing against all other minions.\\
\indent 2.  If the results of step 1 are $ \geq \lfloor \frac{n}{2} \rfloor$ 'both good' results, then $M'$ is a good minion.  The run-time here is $O(n).$  \\
\indent  3.  If $ M' $ did not receive $ \geq \lfloor \frac{n}{2} \rfloor$ 'both good' results, we know that any minion that voted $M'$ as bad is potentially good, and any minion who voted $M'$ as good is bad, and that $M'$ itself is bad.  \\
\indent  Step 3 is also $O(n)$.  The maximum number of times this process can repeat is $\lfloor \frac{n}{2} \rfloor$, since good minions outnumber bad.  Therefore, this process is $\Omega (n)$.\\
\\
6. ?? :(\\
\\
7.  def Dijkstra(V,E): \#from Dasgupta, Papadimitriou, and Vazirani p.115\\
\indent for all $u \in$ V:\\
\indent dist(u) = $\infty$  \\
\indent parent(u) = nil\\
\indent root = $v_0$\\
\indent dist(root) = 0\\
\indent Q = queue()\\
\indent while Q not empty:\\
\indent\indent u = deletemin(Q)\\
\indent\indent for all edges (u,v) $ \in $ E:\\
\indent\indent\indent if v.dist $>$  dist(u) + l(uv):\\
\indent\indent\indent\indent dist(v) = dist(u) + l(u,v)\\
\indent\indent\indent\indent parent(v)=u\\
\indent\indent\indent\indent decreaseKey(Q,v)\\
\\
$G'$ = reverse(G) \# make a copy of graph G with edges reversed\\
Dijkstra(G.V, G.E)\\
Dijkstra(G'.V,G'.E)\\
index1=0\\
index2=0\\
Array[sizeG][sizeG] \# a 2D array to store distance between vertex pairs\\
for v in G:\\
\indent for v' in G':\\
\indent\indent Array[index][index2] = v.distance + v'.distance \\ 
\indent\indent index2 +=1\\
\indent index1 +=1\\
\\
8.  def modifiedDijkstra(V,E,startNode): \#modified from Dasgupta, Papadimitriou, and Vazirani\\
\indent for all $u \in$ V:\\
\indent totalCost(u) = $\infty$  \\
\indent parent(u) = nil\\
\indent root = $startNode$\\
\indent totalCost(root) = 0\\
\indent Q = queue()\\
\indent while Q not empty:\\
\indent\indent u = deletemin(Q)\\
\indent\indent for all edges (u,v) $ \in $ E:\\
\indent\indent\indent if v.totalCost $>$  totalCost(u) + l(uv) + cost(v):\\
\indent\indent\indent\indent totalCost(v) = totalCost(u) + l(u,v) + cost(v)\\
\indent\indent\indent\indent u.CostArray(v) = totalCost(v) \# record the cost from start node to\\
\indent\indent\indent\indent\indent\indent just-closed node, as an attribute of start node\\
\indent\indent\indent\indent parent(v)=u\\
\indent\indent\indent\indent decreaseKey(Q,v)\\
\\
\indent for all v in G:\\
\indent\indent modifiedDijkstra(G.V,G.E,v) \# calculate total cost from u to v for all(u,v) $\in$ G\\





\end{document}  